%%%%%%%%%%%%%%%%%%%%%%%%%%%%%%%%%%%%%%%%%%%%%%%%%%%%%%%%%%%

%% document class
\documentclass[a4paper]{book}

%% packages
\input{settings/packages}

%% page settings
\input{settings/page}
%%%%%%%%%%%%%%%%%%%%%%%%%%%%%%%%%%%%%%%%%%%%%%%%%%%%%%%%%%%
%\newcommand{\imginput}[1]{\input{#1}} %this command is to prevent pdf_latex imgs from showing up in structure
\begin{document}

	
	\frontmatter
	% !TeX root = main
%%%%%%%%%%%%%%%%%%%%%%%%%%%%%%%%%%%%%%%%%
% Minimalist Book Title Page 
% LaTeX Template
% Version 1.0 (27/12/12)
%
% This template has been downloaded from:
% http://www.LaTeXTemplates.com
%
% Original author:
% Peter Wilson (herries.press@earthlink.net)
%
% License:
% CC BY-NC-SA 3.0 (http://creativecommons.org/licenses/by-nc-sa/3.0/)
% 
% Instructions for using this template:
% This title page compiles as is. If you wish to include this title page in 
% another document, you will need to copy everything before 
% \begin{document} into the preamble of your document. The title page is
% then included using \titleTH within your document.
%
%%%%%%%%%%%%%%%%%%%%%%%%%%%%%%%%%%%%%%%%%

\begin{titlepage}
\newenvironment{bottompar}{\par\vspace*{\fill}}{\clearpage}

\raggedleft % Right-align all text
\vspace*{\baselineskip} % Whitespace at the top of the page

{\Large Johnson Anh Huy}\\[0.167\textheight] % Author name

{\LARGE\bfseries A Universally Friendly Resource for General Research}\\[\baselineskip] % First part of the title, if it is unimportant consider making the font size smaller to accentuate the main title

{\color{red}{\Huge Cavity Enhanced Absorption Spectroscopy and Gaussian Beam Profiler }}\\[\baselineskip] % Main title which draws the focus of the reader

\vfill % Whitespace between the title block and the publisher

\vspace*{3\baselineskip} % Whitespace at the bottom of the page
\flushleft
\end{titlepage}

	
	\chapter{Preface}

		Currently, much of the material being taught today in university and college class are extremely out of touch with what is relevant with today's current research, particularly in physics. This has made it difficult for students nearing the end of their studies to jump into todays research projects without getting completely lost. Particular in physics and physical chemistry, research projects have become technologically and theoretically much more complicated and rigourous. Long gone are the days of throwing in material and hoping for the best, boring titrations or smoothening metal surfaces by hand, etc etc. This book is aimed at helping young university students jump into laser spectroscopy to aid in learning the theoretical and technical skills required. This book is just a band-aid solution however and will provide as a base for my long-term future goal of updating the material being taught in elementary school all the way to the university/college level. 
		
		Currently, there is a large gap in experience and knowledge between todays professors and students, particularly again in physics. This large gap has made it difficult for professors to effectively communicate complicated concepts and techniques to students. Much of today's papers are also targeted at todays professors completely allienating todays students which is not helping this problem at all. 
		
		An examples of very complicated yet critical technique for many laser cavity setups is the infamouse Pound Drever Hall technique. Many physics students get completely lost with this technique because knowledge of feedback control theory, modulation and heterodyning is absolutely essential but is not being taught to many students at any point during their studies or at the very least, being given as pre-reading. Professors have essentially progressed many fields, not just laser spectroscopy or cold molecules, significantly and thus actually do not know how to effectively help and guide students in learning complex experiments. This book should explain concepts not currently explained to students from their studies so that they may understand this complex field. This is by no means easy and there is a reason why the content taught today is outdated.
		
		Formerly being one of those poor lost students and now one of the leaders of the field (I solved the cavity problem,created a new cavity aligning technique, and created a systematic method of aligning laser cavity in the infrared red regeion =)  while  still in my early 20s) this book currently serves somewhat as a personal journal and a source at picking at my brain to help communicate everything I needed to learn in order to understand this complex interdisciplinary field to those who will be in the same shoes as I was. I have decided to make this book free on my git repository so that anyone can freely view this book. I also will make a jupyter notebook page to accompany this book for interactivity with this book as some concepts are just better explain with interactive plots etc etc. The jupyter notebook will also serve to help young chemists and physicists learn python as I will comment the crap out of everything.
		
		At some point, I also would like python (more broadly programming in general) to be taught to school people as programming as proven to be a valuable tool in physics. Due to Python's resources being open source, easy to use and learn syntax, generous funding of various packages and projects, it has proven to be a high class, flexible, powerful and universal tool in many fields. Being a chemist, physicist and soon engineer, I will just show be showcasing its uses in these fields but of it course has applications in statistic, finance, mathematics, and data science in general etc etc.
		
		I still have much to learn though so this book(formerly a thesis) will continue to grow as I work on my PhD. A lot of this content will also end up in my more professional and official thesis that I hand for my graduate studies. I admit, I have torrented and never financialy contributed anything during my studies so this will serve as my way of giving back to the academic community. I hope you old farts are ready.

		I will clean up the language when things are finalized, I still got 6 more years till I become a professor. Made on Sunday, December 4, 2016 on Ubuntu 16.10 from my parents house (yeah I'm a currently a bum)
		
		\mainmatter
\end{document}

